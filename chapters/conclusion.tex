\chapter{Conclusion}

    \paragraph*{}
    Due to the growing population, the growth of private vehicles is not going to halt and in turn the demand for parking. And increasing the parking supply by creating more and more parking spaces is not a feasible solution. Because land resources are limited, it isn’t possible to conveniently plan parking spaces according to demand. Instead of increasing available parking spaces, an effective technology-based solution must be employed to optimize the use of available spaces and to  provide drivers with a real-time map of available spaces and allow drivers to reserve spots based on their convenience. Not only does this save drivers time spent searching for a spot, but it also reduces environmental degradation resulting from congestion caused by parking. Finally, the parking system plays a key role in the metropolitan traffic system, and lacking it shows closed relations with traffic congestion, traffic accidents, and environmental pollution. And even with all of these factors, the parking problem is an often-overlooked aspect of urban planning and transportation. Therefore an efficient parking system can improve urban transportation and city environment besides raising the quality of life for citizens.