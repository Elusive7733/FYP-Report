\chapter{Introduction}

    \section{Background}
    \paragraph*{}
    Commuting is a daily part of most people's lives and for many people who own a private vehicle finding a suitable and appropriate parking spot is a daily struggle. It becomes even more difficult to find a spot when hundreds of people come out for an event or holiday festivals. The fact that no private vehicle is perpetually in motion; most private vehicles spend most of their time at rest, either during working hours or over the night, means that there should be two places for every car in the city to be parked in. The two places should be at the both ends of every trip. However the number of vehicles keeps increasing, while parking space has remained constant or reduced due to a growing population.This imbalance between parking supply and parking demand has been considered as the main reason for metropolis parking problems.

    \section{Problem}
    \paragraph*{}
    Lack of parking spots truly is a global issue. As global living standards rise and urbanization accelerates, especially in India and China, cities around the world are seeing huge spikes in motor vehicle ownership accompanied by demand for parking. In India, the number of private automobiles grew nearly 400\% between 2001 and 2015, going from 55 million to 210 million. In China, as of 2017, there was a shortage of 50 million parking spaces, according to the central government.Cities face immense challenges from climate change and rising heat, increased urbanization and housing affordability.

    \paragraph*{}
    Besides the lack of parking spots, there is another problem that needs to be addressed.\\ Surprisingly, there are metropolitan cities which incorporate plenty of parking spaces.
    For example, one study shows that there are 2.2 million total parking spaces in Philadelphia and 1.85 million in New York City. So the problem is not just about the general lack of parking spaces but the lack of real-time data about vacant ones. Due to this, drivers are often left frustrated and spend too much time searching for a spot. In peak hours drivers in large cities have to circle around the desired destination to find a place where they can leave their vehicles. Those who run out of time might park illegally. By bouncing between parking spaces that are full, drivers become restless and may choose to park illegally or leave altogether. 

    \paragraph*{}
    An unregulated tariff structure is another issue which leads to a scarcity of parking spaces. For example, in Indian and Pakistani metros, parking is either free or minimally priced, the fees being unregulated for many years now. Because parking prices stop increasing after a certain period of time, the longer one stays in a parking space, the less one has to pay. This is a problem because parking space is a scarce commodity today and should come with a price. A low parking price encourages more vehicles on the road, contributing to air and noise pollution.

    \section{Proposed System}
    \paragraph*{}
    The project aims to solve all the problems addressed above by attempting to create a self-managing peer to peer parking system where people who have unused parking spots can both earn and contribute by making their spots available for others to park and charge a fare for the spot. The system will embody a map that provides reliable, real-time data that allows drivers to choose from the available spots or even pre-book a parking spot beforehand.


    \section{Pricing}
    \paragraph*{}
    The system will further implement a regulated tariff system. One could say the best way to manage the parking is by charging the right price for it. This can be done by using demand to price parking and optimize occupancy. If the price is too high and spaces remain vacant, operators lose revenue, nearby shops lose customers, employees lose jobs, and governments lose tax revenue. If the price is too low and no spaces are available, it leads to traffic congestion and chaos. Pricing can thus be a very effective tool for the management of travel demand as a whole. Deploying a cost-effective parking management and guidance solution ultimately generates more revenue for a city, as existing parking spaces are properly monetized. Drivers are more motivated to pay for a spot when they know they’ll be able to find it quickly, without having to circle around in vain. The awareness by drivers that all spaces are monitored by a modern system further increases the understanding that it is fair to pay for the valuable public space and service. 

\section{Conclusion}
\paragraph*{}
    Due to the growing population, the growth of private vehicles is not going to halt and in turn the demand for parking. And increasing the parking supply by creating more and more parking spaces is not a feasible solution. Because land resources are limited, it isn’t possible to conveniently plan parking spaces according to demand. Instead of increasing available parking spaces, an effective technology-based solution must be employed to optimize the use of available spaces and to  provide drivers with a real-time map of available spaces and allow drivers to reserve spots based on their convenience. Not only does this save drivers time spent searching for a spot, but it also reduces environmental degradation resulting from congestion caused by parking. Finally, the parking system plays a key role in the metropolitan traffic system, and lacking it shows closed relations with traffic congestion, traffic accidents, and environmental pollution. And even with all of these factors, the parking problem is an often-overlooked aspect of urban planning and transportation. Therefore an efficient parking system can improve urban transportation and city environment besides raising the quality of life for citizens.
